\documentclass[a4paper,12pt]{article}
\usepackage[utf8]{inputenc}
\usepackage{graphicx}
\usepackage{amsmath}
\usepackage{hyperref}
\usepackage{float}

\title{Predição de Pacientes com Diabetes}
\author{Pedro Jorge de Souza Colombrino \\ Matheus Ferreira Amaral Madeira \\ Guilherme Vieira Rodrigues }
\date{\today}

\begin{document}

\maketitle

\begin{abstract}
    Neste projeto, abordamos a previsão de vendas utilizando técnicas de \textbf{regressão linear} e o modelo \textbf{ARIMA} (AutoRegressive Integrated Moving Average). O conjunto de dados original, denominado \textbf{VendasMensais}, contém informações sobre vendas mensais, incluindo receita, quantidade de vendas, custo médio e folha de pagamento anual da região. Este conjunto de dados foi obtido a partir de uma fonte confiável e é fundamental para a análise de desempenho de vendas ao longo do tempo.
\end{abstract}

\section{Introdução}

A primeira etapa do projeto foi a análise exploratória dos dados (EDA), realizada no notebook \textbf{EDA.ipynb}. Durante essa fase, o conjunto de dados foi carregado e tratado, onde foram identificados e manipulados dados faltantes, outliers e a formatação da coluna de datas. O resultado desse tratamento foi um novo arquivo CSV, \textbf{VendasMensais\_processed.csv}, que apresenta dados limpos e prontos para análise.

O modelo ARIMA foi escolhido devido à sua eficácia em lidar com séries temporais, especialmente quando se busca prever valores futuros com base em dados históricos. O uso do \textbf{auto\_arima} permitiu a seleção automática dos melhores parâmetros do modelo, facilitando o processo de modelagem e melhorando a precisão das previsões. O ARIMA é vantajoso por sua capacidade de capturar padrões sazonais e tendências nos dados, o que é crucial para a previsão de vendas.

Após a aplicação do modelo ARIMA, foram obtidos resultados promissores, com métricas de precisão como RMSE, MAE e R², que indicam a qualidade das previsões em relação aos dados reais. Os gráficos gerados demonstraram a comparação entre os dados históricos e as previsões futuras, permitindo uma visualização clara do desempenho do modelo.

\section{Metodologia}

\subsection{Tratamento de Dados}

O tratamento dos dados foi realizado em várias etapas, incluindo:

\begin{itemize}
    \item \textbf{Identificação de Dados Faltantes}: Linhas com valores ausentes foram analisadas e tratadas para garantir a integridade dos dados;
    \item \textbf{Detecção de Outliers}: Valores extremos foram identificados e avaliados para determinar se deveriam ser removidos ou ajustados;
    \item \textbf{Formatação de Datas}: A coluna de datas foi convertida para o formato datetime, permitindo uma análise temporal adequada.
\end{itemize}

\subsection{Modelo ARIMA}

O modelo ARIMA foi implementado utilizando a biblioteca \textbf{pmdarima}. O processo incluiu:

\begin{itemize}
    \item \textbf{Definição de Variáveis}: A variável de interesse, receita, foi selecionada para a modelagem;
    \item \textbf{Treinamento do Modelo}: O modelo foi treinado utilizando a função \textbf{auto\_arima}, que automaticamente seleciona os melhores parâmetros para o modelo ARIMA;
    \item \textbf{Previsão}: O modelo foi utilizado para prever as receitas futuras para os próximos 24 meses.
\end{itemize}

\section{Resultados}
\subsection{Explicativas}
Conforme proposto na atividade, as previsões deveriam abranger os anos de 2024 e 2025. No entanto, devido à indisponibilidade de dados suficientes para realizar previsões precisas para essas datas, mantivemos o horizonte de previsão original. Ajustamos a linha temporal para garantir que as previsões sejam coerentes com os dados disponíveis, permitindo uma análise mais precisa e relevante.


\begin{table}[H]
    \centering
    \caption{Previsões de Receita para os Próximos Meses}
    \begin{tabular}{|c|c|}
        \hline
        \textbf{Data} & \textbf{Predicted Revenue} \\
        \hline
        2020-05-01 & 38831.81 \\
        2020-06-01 & 54009.85 \\
        2020-07-01 & 52292.29 \\
        2020-08-01 & 43233.82 \\
        2020-09-01 & 54826.41 \\
        2020-10-01 & 50460.57 \\
        2020-11-01 & 42675.11 \\
        2020-12-01 & 65056.62 \\
        2021-01-01 & 62588.45 \\
        2021-02-01 & 46525.39 \\
        2021-03-01 & 56322.32 \\
        2021-04-01 & 58620.84 \\
        2021-05-01 & 45131.97 \\
        2021-06-01 & 60310.01 \\
        2021-07-01 & 58592.45 \\
        2021-08-01 & 49533.98 \\
        2021-09-01 & 61126.57 \\
        2021-10-01 & 56760.73 \\
        2021-11-01 & 48975.27 \\
        2021-12-01 & 71356.78 \\
        2022-01-01 & 68888.61 \\
        2022-02-01 & 52825.55 \\
        2022-03-01 & 62622.48 \\
        2022-04-01 & 64920.99 \\
        \hline
    \end{tabular}
\end{table}

As métricas de desempenho do modelo foram calculadas e apresentadas a seguir:

\begin{itemize}
    \item \textbf{RMSE}: 15064.31
    \item \textbf{MAE}: 14131.92
    \item \textbf{R²}: -1.79
\end{itemize}

A métrica RMSE (Root Mean Square Error) de 15064.31 indica que, em média, as previsões do modelo estão a cerca de 15064.31 unidades de receita do valor real. Um RMSE mais baixo é desejável, pois indica uma melhor precisão nas previsões. 

O MAE (Mean Absolute Error) de 14131.92 complementa essa análise, mostrando que, em média, as previsões estão a 14131.92 unidades de receita do valor real, sem considerar a direção do erro.

Por outro lado, o valor de R² de -1.79 sugere que o modelo não se ajustou bem aos dados. O R² é uma medida que indica a proporção da variabilidade dos dados que é explicada pelo modelo. Um valor negativo indica que o modelo é pior do que uma média simples dos dados, o que é preocupante e sugere que o modelo ARIMA pode não ser o mais adequado para esta série temporal específica.

\section{Possíveis Melhorias}

Para obter melhores resultados em projetos futuros, algumas abordagens podem ser consideradas:

\begin{itemize}
    \item \textbf{Exploração de Outros Modelos}: Testar diferentes modelos de previsão, como modelos de suavização exponencial, modelos de regressão ou redes neurais, podem nos ajudar a encontrar uma abordagem mais adequada para os dados;
    \item \textbf{Ajuste de Parâmetros}: Realizar uma busca mais abrangente por hiperparâmetros, utilizando técnicas como validação cruzada, pode melhorar a performance do modelo;
    \item \textbf{Análise de Variáveis Externas}: Incluir variáveis exógenas que possam influenciar as vendas, como campanhas de marketing, sazonalidade ou eventos econômicos, pode ajudar a capturar melhor a dinâmica das vendas.
\end{itemize}

Essas melhorias podem contribuir para um desempenho mais robusto e confiável em previsões futuras, permitindo que a empresa tome decisões mais informadas com base em dados históricos.

\begin{figure}[H]
    \centering
    \includegraphics[width=1.0\textwidth]{Arima_temporal_series.png} % Substitua pelo caminho do seu gráfico
    \caption{Comparação entre Dados Históricos e Previsões Futuras}
    \label{fig:comparacao}
\end{figure}

\section{Possíveis Falhas ao obter o código python:}
O código em python e todo o percorrer do projeto foram anexados na atividade, em caso de falhas técnicas no recebimento dos arquivos, disponibilizamos aqui o clicável para o \href{https://github.com/pedrojorge1559/Linear-Regression-for-Sales}{\textbf{GitHub}} do projeto.
\newline E em link extenso: https://github.com/pedrojorge1559/Linear-Regression-for-Sales
\end{document}
