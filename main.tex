\documentclass[a4paper,12pt]{article}
\usepackage[utf8]{inputenc}
\usepackage{graphicx}
\usepackage{amsmath}
\usepackage{hyperref}
\usepackage{float}
\usepackage{geometry}

\geometry{top=2cm, bottom=2cm, left=2cm, right=2cm}

\title{Uso de Algoritmos de Classificação para Predição de Novos Casos de Diabetes Mellitus}
\author{Pedro Jorge de Souza Colombrino \\ Matheus Ferreira Amaral Madeira \\ Guilherme Vieira Rodrigues}
\date{31 de Outubro, 2024}

\begin{document}

\maketitle

\begin{abstract}
Este estudo apresenta a aplicação de algoritmos de aprendizado de máquina, com foco em \textit{RandomForestClassifier}, para a previsão de novos casos de diabetes em mulheres. O objetivo principal é demonstrar como treinar, avaliar e implementar esses modelos na predição de uma variável dependente dicotômica (\textit{diabético}) com base em dados clínicos e demográficos. Utilizando validação cruzada (\textit{K-Fold}), identificamos o modelo mais eficaz e avaliamos sua aplicabilidade em cenários reais de saúde pública.
\end{abstract}

\section{Introdução}

A capacidade de prever novos casos de diabetes com base em dados clínicos é uma ferramenta valiosa para a saúde pública e medicina personalizada. Diabetes Mellitus é uma doença crônica associada a sérios riscos de complicações, sendo prevalente em mulheres devido a fatores como diabetes gestacional e alterações hormonais.

Com a crescente disponibilidade de dados clínicos, algoritmos de aprendizado de máquina destacam-se pela capacidade de identificar padrões complexos e fazer previsões precisas. Este trabalho explora a aplicação do algoritmo \textit{RandomForestClassifier} para prever diabetes com base em características clínicas e demográficas, demonstrando como os modelos podem ser aplicados na prática clínica.

\section{Metodologia}

\subsection{Conjunto de Dados}
Os dados utilizados foram obtidos do Kaggle, com 768 registros de pacientes do sexo feminino, contendo 8 variáveis explicativas e uma variável alvo (\textit{Diabético}) que indica a presença ou ausência de diabetes.

\begin{table}[H]
    \centering
    \caption{Descrição das Variáveis do Conjunto de Dados}
    \begin{tabular}{|p{3cm}|p{8cm}|p{3cm}|}
        \hline
        \textbf{Variável} & \textbf{Descrição} & \textbf{Tipo de Dado} \\
        \hline
        Gravidez & Número de gestações do paciente & Inteiro \\
        \hline
        Glicose & Concentração de glicose no plasma após teste oral & Inteiro \\
        \hline
        PressaoSanguinea & Pressão arterial diastólica (mmHg) & Inteiro \\
        \hline
        EspessuraDaPele & Espessura da dobra cutânea tricipital (mm) & Inteiro \\
        \hline
        Insulina & Nível sérico de insulina de 2h (mu U/ml) & Inteiro \\
        \hline
        IMC & Índice de Massa Corporal (peso em kg/(altura em m)$^2$) & Decimal \\
        \hline
        DiabetesPedigree & Histórico genético de diabetes & Decimal \\
        \hline
        Idade & Idade do paciente (anos) & Inteiro \\
        \hline
        Diabético & Presença de diabetes (1: Sim, 0: Não) & Inteiro \\
        \hline
    \end{tabular}
\end{table}

\subsection{Modelo: RandomForestClassifier}
\textit{RandomForestClassifier} é uma abordagem baseada em árvore de decisão que usa um conjunto de árvores para melhorar a robustez e a precisão. É adequado para detectar interações não lineares entre variáveis.

\subsection{Validação Cruzada com K-Fold}
\textit{K-Fold} A validação cruzada foi usada para avaliar o desempenho do modelo. O conjunto de dados foi dividido em 5 subconjuntos (\textit{folds}), garantindo que todos os dados fossem utilizados para treinamento e teste em diferentes iterações.

\begin{itemize}
    \item Reduz a possibilidade de \textit{overfitting}.
    \item fornece métricas de desempenho mais confiáveis.
    \item garante uma avaliação mais abrangente do modelo.
\end{itemize}
Para as features da ferramenta \textit{K-Fold}, utilizamos de ténicas heurísticas para uma aproximação prática, onde:
\begin{equation}
    n\_splits = \min(5, \lfloor \frac{n}{10} \rfloor)
    \end{equation}
\( n \) é o número total de amostras.
\newline
\newline Também utilizamos a feature \textit{Shuffle}, que aceita entradas booleanas, definido como \textit{True} e \textit{random\_state} definido em 42, garantindo uma grande representatividade dos dados e uma boa reprodutibilidade dos resultados.

\section{Treinamento e Avaliação}

\subsection{Hiperparâmetros do RandomForestClassifier}
Os hiperparâmetros de \textit{RandomForestClassifier} controlam o comportamento e o desempenho do modelo. Aqui, adotamos uma configuração baseada em avaliação heurística, que é uma aproximação prática do problema real e é particularmente útil quando não há dados suficientes para uma busca exaustiva de hiperparâmetros.

\begin{itemize}
    \item \textbf{Número de árvores (\texttt{n\_estimators}):} Define o número de árvores na floresta. Um número maior pode melhorar a estabilidade do modelo, mas por outro lado aumenta o custo computacional. A fórmula heurística utilizada é:
    \begin{equation}
    n\_estimators = 10 \times \sqrt{n},
    \end{equation}
    onde \( n \) é o número total de amostras. Este cálculo fornece um ponto de partida para um equilíbrio entre desempenho e eficiência computacional.

    \item \textbf{Profundidade Máxima (\texttt{max\_depht}):} Este hiperparâmetro controla a profundidade em que cada árvore pode crescer. Árvores mais profundas tendem a capturar mais detalhes, mas podem levar ao \textit{overfitting}. A profundidade é limitada pela seguinte fórmula:
    \begin{equation}
    max\_depth = \log_2(n),
    \end{equation}
    onde \( n \) é o número total de amostras. Isso permite capturar padrões importantes sem complicar demais o modelo.

    \item \textbf{Número mínimo de amostras para divisão (\texttt{min\_samples\_split}):} Define o número mínimo de amostras necessárias para dividir um nó. Para evitar partições muito pequenas e garantir robustez usamos:
    \begin{equation}
    min\_samples\_split = \max(2, \frac{n}{100}),
    \end{equation}
    onde \( n \) é o número total de amostras.

    \item \textbf{Amostras mínimas por folha (\texttt{min\_samples\_leaf}):} Determina o número mínimo de amostras permitido em uma folha terminal. A heurística utilizada é:
    \begin{equation}
    min\_samples\_leaf = \max(1, \frac{n}{1000}),
    \end{equation}
    garantindo que cada folha tenha um número mínimo de amostras para conclusões confiáveis.

    \item \textbf{Número máximo de atributos por Divisão (\texttt{max\_features}):} Define quantos atributos são considerados para encontrar a melhor partição. Para problemas de classificação, as escolhas comuns são:
    \begin{equation}
    max\_features = \sqrt{m},
    \end{equation}
    onde \( m \) é o número total de atributos no conjunto de dados.
\end{itemize}

Essas heurísticas fornecem uma base sólida para configurar o modelo antes de realizar ajustes mais avançados, como otimização de hiperparâmetros via \textit{Grid Search} ou \textit{Random Search}. O uso destas avaliações iniciais é amplamente aceito na prática porque proporciona um equilíbrio entre simplicidade e desempenho.

\subsection{Métricas de Avaliação}
O desempenho foi avaliado com a métrica de \textit{acurácia}, calculada como:
\[
\text{Acurácia} = \frac{\text{Número de Previsões Corretas}}{\text{Número Total de Previsões}}
\]

\section{Resultados}
\singlespacing

Os resultados da validação cruzada mostram que a precisão média por fold é a seguinte:

\begin{itemize}
    \item Precisão do modelo da 1ª fold: 0,74
    \item Precisão do modelo da 2ª fold: 0,79
    \item Precisão do modelo na 3ª fold: 0,79
    \item Precisão do modelo de 4ª fold: 0,78
    \item Precisão do modelo de 5ª fold: 0,74
\end{itemize}

A melhor precisão do modelo salvo foi de 0,79.
Estes resultados demonstram que o modelo \textit{RandomForestClassifier} apresenta desempenho consistente e robusto com precisão média geral satisfatória. A validação cruzada confirmou a estabilidade do modelo, com 79\% de acurácia.


\section{Aplicação Prática}

Após treinar o modelo, use a biblioteca \texttt{pickle} para salvar o modelo e depois carregá-lo para prever novos casos. Aqui está um exemplo:

\begin{verbatim}
import pickle

nova_linha = [[5, 176, 72, 17, 24.6, 0.387, 34]]
# Carregar o modelo salvo
with open("melhor_modelo_random_forest.pkl", "rb") as f:
    modelo = pickle.load(f)
# Previsão
predicao = modelo.predict(nova_linha)
\end{verbatim}

O resultado indica se o paciente é diabético (\texttt{1}) ou não (\texttt{0}).
\vspace{1em}

O uso da biblioteca \texttt{pickle} para salvar o modelo treinado permite que ele seja carregado em qualquer máquina sem a necessidade de reprocessar os dados ou reler o CSV original. Isso facilita a implementação do modelo em ambientes de produção, onde previsões precisam ser feitas rapidamente e com eficiência. Basta carregar o arquivo do modelo salvo e utilizá-lo para prever novos casos, garantindo que o processo seja ágil e sem a sobrecarga computacional de treinar o modelo novamente.


\section{Conclusão}

O uso de \textit{RandomForestClassifier} demonstrou ser eficaz na previsão de diabetes em mulheres. Melhorias futuras incluem aumentar o conjunto de dados e aplicar técnicas de otimização de hiperparâmetros.

\end{document}
