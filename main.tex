\documentclass[a4paper,12pt]{article}
\usepackage[utf8]{inputenc}
\usepackage{graphicx}
\usepackage{amsmath}
\usepackage{hyperref}
\usepackage{float}

\title{Uso de Algoritmos de Classificação para Predição de Novos Casos de Diabetes Mellitus}
\author{Pedro Jorge de Souza Colombrino \\ Matheus Ferreira Amaral Madeira \\ Guilherme Vieira Rodrigues}
\date{31 de Outubro, 2024}

\begin{document}

\maketitle

\begin{abstract}
Este estudo apresenta a aplicação de algoritmos de aprendizagem
Máquina, focada em RandomForestClassifier, para previsão de novos casos de diabetes em mulheres. O objetivo é demonstrar como treinar, avaliar e usar esses modelos para prever a variável dependente diabetes (dicotômica) com base em novas entradas de dados clínicos e demográficos. Usando verificação cruzada através do K-Fold, procuramos identificar os modelos mais eficazes e validar a sua aplicação prática em cenários reais de saúde pública.
\end{abstract}

\section{Introdução}

A capacidade de prever novos casos de diabetes com base em dados clínicos é uma ferramenta poderosa para a saúde pública e a medicina personalizada. A diabetes é uma doença crónica associada a graves riscos de complicações e morte, sendo particularmente prevalente nas mulheres devido a fatores específicos como a diabetes gestacional e alterações hormonais. 

À medida que a disponibilidade de dados clínicos continua a aumentar, os algoritmos de aprendizagem automática destacam-se pela sua capacidade de identificar padrões complexos e fazer previsões precisas. Este estudo explora a aplicação de algoritmos de classificação, com foco em \textit{RandomForestClassifier}, para prever novos casos de diabetes em mulheres com base em suas características clínicas e demográficas.

O foco principal deste trabalho é demonstrar como modelos treinados e validados podem ser usados para prever a presença de diabetes em novas linhas de dados, fornecendo uma abordagem prática e aplicável à tomada de decisão clínica.

\section{Metodologia}

\subsection{Conjunto de Dados}
O conjunto de dados utilizado foi do Kaggle e continha informações sobre 768 pacientes do sexo feminino. Cada registro é composto por 8 variáveis clínicas e demográficas e uma variável alvo (\textit{Diabetic}) que indica a presença ou ausência de diabetes.

\begin{table}[H]
    \centering
    \caption{Dicionário de Dados}
    \begin{tabular}{|p{3cm}|p{7cm}|p{3cm}|}
        \hline
        \textbf{Variável} & \textbf{Descrição} & \textbf{Tipo de dado} \\
        \hline
        Gravidez & Número de vezes que o paciente esteve grávido & int \\
        \hline
        Glicose & Concentração de glucose no plasma após um teste oral de tolerância à glucose de 2 horas & int \\
        \hline
        PressaoSanguinea & Pressão arterial diastólica (mm Hg) & int \\
        \hline
        EspessuraDaPele & Dobra cutânea tricipital (mm) & int \\
        \hline
        Insulina & Insulina sérica de 2 horas (mu U/ml) & int \\
        \hline
        IMC & Índice de massa corporal (peso em kg/(altura em m)^2) & float \\
        \hline
        DiabetesPedigree & Função que representa o pedigree da diabetes do paciente & float \\
        \hline
        Idade & Idade do paciente (anos) & int \\
        \hline
        Diabetico & Resultado binário (0 ou 1) em que 1 indica a presença de diabetes & int \\
        \hline
    \end{tabular}
\end{table}

\subsection{Algoritmo de Classificação: RandomForestClassifier}
Neste trabalho, utilizamos \textit{RandomForestClassifier}, um modelo baseado em árvore de decisão que combina múltiplas árvores para gerar previsões robustas. Sua capacidade de lidar com variáveis complexas e detectar interações não lineares o torna ideal para resolver tais problemas.

\subsection{Processo de Validação}
A validação do modelo foi realizada utilizando a técnica de validação cruzada K-fold (5 vezes). Esta abordagem:
\begin{itemize}
    \item Garante que o modelo seja avaliado em relação a todos os dados disponíveis;
    \item Reduz o risco de overfitting;
    \item Fornece métricas de desempenho mais consistentes e confiáveis.
\end{itemize}

\subsection{Critério para Seleção do Melhor Modelo}
Ao final do processo de validação cruzada, o modelo com maior precisão média em \textit{folds} é selecionado e salvo em disco usando a biblioteca \texttt{pickle}. O modelo é posteriormente usado para prever novas entradas de dados.

\section{Uso do Modelo Treinado para Predição de Novos Casos}

Depois de selecionar o melhor modelo, usamos o \textit{RandomForestClassifier} salvo para fazer previsões sobre novos dados. Este processo demonstra a aplicabilidade prática do modelo em cenários reais.

\subsection{Exemplo de Aplicação}

Considere os dados de um novo paciente:

\begin{verbatim}
    novo_paciente = [[5, 176, 72, 17, 24.6, 0.387, 34]]
\end{verbatim}

Os passos para realizar a previsão são:
\begin{enumerate}
    \item Carregar o modelo salvo utilizando a biblioteca \texttt{pickle}.
    \item Passar os dados do novo paciente ao modelo para gerar a predição.
    \item Interpretar o resultado: 1 (diabético) ou 0 (não diabético).
\end{enumerate}

\begin{verbatim}
    # Carregar o modelo salvo
    with open("melhor_modelo_random_forest.pkl", "rb") as f:
        modelo = pickle.load(f)

    # Prever o novo paciente
    predicao = modelo.predict(novo_paciente)

    # Resultado
    resultado = "Diabético" if predicao[0] == 1 else "Não diabético"
    print(f"Previsão: {resultado}")
\end{verbatim}

\section{Resultados}

Os resultados da validação cruzada mostram que \textit{RandomForestClassifier} tem uma precisão média de 85\% nos dados de teste, destacando sua eficiência na previsão de novas situações. O modelo demonstra robustez e consistência globais.

\section{Conclusão}

Este estudo demonstra a eficácia do uso de algoritmos de classificação (com ênfase em \textit{RandomForestClassifier}) para prever novos casos de diabetes em mulheres. As aplicações práticas do modelo, exemplificadas pelas previsões sobre novos dados, destacam o potencial destas ferramentas para apoiar a tomada de decisões clínicas e estratégias de saúde pública. Melhorias futuras incluem a adição de mais dados e o uso de técnicas de otimização de hiperparâmetros para melhorar ainda mais o desempenho do modelo.

\end{document}
